%------------------------
% Resume Template
% Author : Anubhav Singh
% Github : https://github.com/xprilion
% License : MIT
%------------------------

\documentclass[a4paper,20pt]{article}

\usepackage{latexsym}
\usepackage[empty]{fullpage}
\usepackage{titlesec}
\usepackage{marvosym}
\usepackage[usenames,dvipsnames]{color}
\usepackage{verbatim}
\usepackage{enumitem}
\usepackage[pdftex]{hyperref}
\usepackage{fancyhdr}
\usepackage{etaremune} %reverse enumeration

\pagestyle{fancy}
\fancyhf{} % clear all header and footer fields
\fancyfoot{}
\renewcommand{\headrulewidth}{0pt}
\renewcommand{\footrulewidth}{0pt}

% Adjust margins
\addtolength{\oddsidemargin}{-0.530in}
\addtolength{\evensidemargin}{-0.375in}
\addtolength{\textwidth}{1in}
\addtolength{\topmargin}{-.45in}
\addtolength{\textheight}{1in}

\urlstyle{rm}

\raggedbottom
\raggedright
\setlength{\tabcolsep}{0in}

% Sections formatting
\titleformat{\section}{
  \vspace{-10pt}\scshape\raggedright\large
}{}{0em}{}[\color{black}\titlerule \vspace{-6pt}]

%-------------------------
% Custom commands
\newcommand{\resumeItem}[2]{
  \item\small{
    \textbf{#1}{: #2 \vspace{-2pt}}
  }
}

\newcommand{\resumeItemWithoutTitle}[1]{
  \item\small{
    {\vspace{-2pt}}
  }
}

\newcommand{\resumeSubheading}[4]{
  \vspace{-1pt}
    \item
    \begin{tabular*}{0.97\textwidth}{l@{\extracolsep{\fill}}r}
      \textbf{#1} & #2 \\
      \textit{#3} & \textit{#4} \\
    \end{tabular*}\vspace{-5pt}
}


\newcommand{\resumeSubItem}[2]{\resumeItem{#1}{#2}\vspace{-3pt}}

\renewcommand{\labelitemii}{$\circ$}

\newcommand{\resumeSubHeadingListStart}{\begin{itemize}[leftmargin=*]}
\newcommand{\resumeSubHeadingListEnd}{\end{itemize}}
\newcommand{\resumeItemListStart}{\begin{itemize}}
\newcommand{\resumeItemListEnd}{\end{itemize}\vspace{-5pt}}

%-----------------------------
%%%%%%  CV STARTS HERE  %%%%%%

\begin{document}

%----------HEADING-----------------
\begin{tabular*}{\textwidth}{l@{\extracolsep{\fill}}r}
  \textbf{{\LARGE Ty Balduf}} & Email: \href{mailto:Tbalduf@gmail.com}{tbalduf@gmail.com}\\
  \href{https://linkedin.com/in/tybalduf/}{LinkedIn: linkedin.com/in/tybalduf/} & 
  \href{https://github.com/TyBalduf}{Github: github.com/TyBalduf} \\
\end{tabular*}

%-----------EDUCATION-----------------
\section{~~Education}
  \resumeSubHeadingListStart
     \resumeSubheading
      {University of Kansas}{Lawrence, Kansas}
      {PhD - Computational Chemistry}{2016 - 2021}
    \resumeSubheading
      {Saint Ambrose University}{Davenport, Iowa}
      {Bachelor of Science - Chemistry/Math (Computer Science Minor)}{2012 - 2016}
    \resumeSubHeadingListEnd
	    
%-----------Awards-----------------
\section{Honors and Awards}
\resumeSubHeadingListStart
	\resumeSubItem{Walrafen Scholar}{~~For Graduate Study in Physical Chemistry  \textit{2016-2021}}
	\resumeSubItem{Bailey Scholar}{~~~~~~For Graduate Study in Chemistry \textit{2016-2021}}
	\resumeSubItem{Amini Scholar}{~~~~~~Outstanding Physical Chemistry Research \textit{2017, 2019, 2021}}
	\resumeSubItem{Travel Grant}{~~~~~~~~Research Excellence Initiative \textit{2018, 2019}}
	\resumeSubItem{Ambrose Scholar}{~~Full Tuition Scholarship \textit{2012-2016}}
    \resumeSubHeadingListEnd

\vspace{-5pt}
\section{Publications}
  \begin{etaremune}
  \item Balduf T.; Caricato M. \textit{Quantum Mechanical Calculation of the Full Optical Rotation Tensor for Periodic Systems} \textbf{In Progress}
  \item Parsons T.; Balduf T.; Caricato M. \textit{Effect of Level of Theory and Gauge on Computed Optical Rotation} \textbf{In Progress}
  \item Balduf T.; Jystad A.; Lynam D.; Hopkins-Lesberg J.;  Henke W.; Blakemore J.; Caricato M. \\ \textit{Computational Insights into Ligand Influences on Hydrogen Generation with [Cp*Rh] Hydrides} \textbf{Submitted} 
  \item Caricato, M.; Balduf T. \textit{Origin Invariant Full Optical Rotation Tensor in the Length Dipole Gauge without London Atomic Orbitals} The Journal of Chemical Physics \textbf{2021}, \textit{155}, 024118 DOI: \href{https://doi.org/10.1063/5.0053450}{10.1063/5.0053450} 
  \item Balduf, T.; Caricato, M. \textit{Gauge Dependence of the $\tilde{S}$ Molecular Orbital Space Decomposition of Optical Rotation} J. Phys. Chem. A \textbf{2021}, \textit{125}, 23, 4976 DOI: \href{https://doi.org/10.1021/acs.jpca.1c01653}{10.1021/acs.jpca.1c01653} 
  \item Zhang, K., Balduf, T., Caricato, M. \textit{Full Optical Rotation Tensor at Coupled Cluster with Single and Double Excitations Level in the Modified Velocity Gauge.} Chirality. \textbf{2021}, \textit{33}, 303. DOI:   \href{https://doi.org/10.1002/chir.23310}{10.1002/chir.23310}
  \item Balduf, T.; Caricato, M. \textit{Helical Chains of Diatomic Molecules as a Model for Solid State Optical Rotation} \\J. Phys. Chem. C, \textbf{2019}, \textit{123}, 4329 DOI: \href{https://doi.org/10.1021/acs.jpcc.8b12084}{10.1021/acs.jpcc.8b12084} 
  \end{etaremune}
\vspace{-5pt}
	    
%-----------PROJECTS-----------------
\vspace{-5pt}
\section{Presentations}
\begin{etaremune}

\item ACS Midwest Regional Meeting, SpringField, MO (Oct. 2021): \textit{Quantum Mechanical Calculation of the Full Optical Rotation Tensor for Periodic Systems} \textbf{Talk}

\item ACS Midwest Regional Meeting, SpringField, MO (Oct. 2021): \textit{Gauge dependence of the $\tilde{S}$ molecular orbital space decomposition of optical rotation} \textbf{Poster}

\item ACS Midwest Regional Meeting, Wichita, KS (Oct. 2019): \textit{Helical Chains of Diatomic Molecules as a Model for Solid State Optical Rotation} \textbf{Talk}

\item International Conference on Chiroptical Spectroscopy, Pisa, Italy (Jun. 2019): \textit{Helical Chains of Diatomic Molecules as a Model for Solid State Optical Rotation} \textbf{Poster} 

\item NSF/DOE Quantum Science Summer School, State College, PA (Jun. 2019): \textit{Helical Chains of Diatomic Molecules as a Model for Solid State Optical Rotation} \textbf{Poster} 

\item St. Ambrose University Chemistry Symposium, Davenport, IA (May 2019): \textit{Helical Chains of Diatomic Molecules as a Model for Solid State Optical Rotation} \textbf{Talk} 

\item Kansas Physical Chemistry Symposium, Manhattan, KS (Sep. 2018): \textit{Optical Rotation of Solids: Tensor Component Analysis of Model Helical Chains of Diatomics} \textbf{Poster} 

\item ACS National Meeting, New Orleans, LA (Mar. 2018): \textit{Optical Rotation of Helical Chains of Diatomics as a Model for Periodic Systems} \textbf{Talk} 

\item ACS Midwest Regional Meeting, University of Kansas, Lawrence, KS (Oct. 2017): \textit{Electronic Coupling in Diazulenic Molecular Rectifiers: A Theoretical Approach} \textbf{Talk} 

\item Undergraduate Summer Research Institute, Davenport, IA (Aug. 2014): \textit{Polystyrene Degradation: Optimal Photosensitization in the Presence of Benzophenone} \textbf{Talk}
\end{etaremune}	    
	    
\vspace{-5pt}
\section{Computational Proficiency}
	\resumeSubHeadingListStart
	\resumeSubItem{Languages}{~~~~~~~~Python, Fortran, Java, Julia}
	\resumeSubItem{Chem Software}{~Gaussian, Psi4, ASE, Avogadro}
	\resumeSubItem{Tools}{~~~~~~~~~~~~~~~~Git, LaTeX, Bash/Zsh/Tcsh}
	\resumeSubItem{Courses}{~~~~~~~~~~~~Programming Language Concepts, Data Structures, Algorithm Analysis}
    \resumeSubHeadingListEnd
    
\vspace{-5pt}
\section{Experience}
  \resumeSubHeadingListStart
    \resumeSubheading{Research Assistant}{Lawrence, KS}
    {Caricato Research Group}{May 2017 - Present}
    \resumeItemListStart
        \resumeItem{Methods Development}
          {Developed new electronic structure methods to facilitate the interpretation and calculation of optical rotation. Implemented these methods into the Gaussian electronic structure program.}
          \resumeItem{Mentoring}
          {Worked directly with graduate (3), undergraduate (2), and high school (1) students to further their research projects. Aided with onboarding for technology/techniques used in the group. Guided the students through experiments and how best to present their findings.}
      \resumeItemListEnd
    \resumeSubheading
		{Teaching Assistant}{Lawrence, KS}
		{Biophysical Chemistry Lab}{Jan. 2017 - May 2017}
		\resumeItemListStart
        \resumeItem{Research Process}
          {Challenged students to think critically about the research process and the implications of experimental/theoretical results}
        \resumeItem{Interdisciplinary}
          {Demonstrated the interplay between theory and experiment in scientific research}
		\resumeItemListEnd

    \resumeSubheading
		{Teaching Assistant}{Lawrence, KS}
		{General Chemistry Lab}{Aug. 2016 - Dec. 2016}
		\resumeItemListStart
        \resumeItem{Problem Solving}
          {Presented how the scientific method and chemical reasoning are applied in interpreting experiments}
        \resumeItem{Technical Skills}
          {Introduced common instrumentation and basic lab technique}
		\resumeItemListEnd
\resumeSubHeadingListEnd


\vspace{-5pt}
\section{Professional Activities and Service}
  \resumeSubHeadingListStart
	\resumeSubItem{GSO Alumni Committee Representative}
    {Subcommittee on Student Recruitment, Development, and Placement. Organized career panel with KU alumni. \textit{2019-2020}}
    \resumeSubItem{KU Carnival of Chemistry Volunteer}
    {Performed demonstrations of scientific experiments for local elementary school students and their families. \textit{2018, 2019}}
    \resumeSubItem{KU Ambrose White Lecture Co-Organizer}
    {Assisted in speaker search and selection. Drafted program for the lecture. \textit{2019}}

\resumeSubHeadingListEnd

\end{document}